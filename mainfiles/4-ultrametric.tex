%!TEX root = ../kursovaya.tex

\section{Невыполнение ультраметрического неравенства}
Для облака $[\Delta_{1}]$, по \ref{remUltraMetric} справедливо
ультраметрическое неравенство.
 Следующая лемма показывает, что для облака $[\mathbb{R}]$
это неравенство может не выполняться.

Рассмотрим $\mathbb{R}$ как подмножество
$\mathbb{R}^2$ и добавим к нему точку $(0,1)$, расстояние до которой будет
соответствовать метрике $L_{1}$ в $\mathbb{R}^2$. Обозначим это
пространство $\widetilde{\mathbb{R}}$.
\begin{theorem}
Для пространств $\mathbb{Z}$ и $\widetilde{\mathbb{R}}$ выполняются следующие
утверждения:
\begin{enumerate}
  \item Пространства $\mathbb{Z}$ и $\widetilde{\mathbb{R}}$ находятся от $\mathbb{R}$
на расстоянии не большем $\frac 1 2$.\label{thrmPt:1}
  \item Расстояние между $\mathbb{Z}$ и $\widetilde{\mathbb{R}}$ строго
больше $\frac 1 2$.\label{thrmPt:2}
\end{enumerate}
\label{thrmRUltraMetric}
\end{theorem}
\begin{proof}
Вложением целых чисел в вещественную прямую получается реализация $\mathbb{Z}$, $\mathbb{R}$ с расстоянием Хаусдорфа равным $\frac 1 2$.
Если вложить $\widetilde{\mathbb{R}}$ в $\mathbb{R}^{2}$ естественным образом, а $\mathbb{R}$ вложить как подмножество равное $\{(x, \frac1 2 )|x\in \mathbb{R}\}$,
расстояние Хаусдорфа между ними также будет равно $\frac 1 2$. Таким образом, доказано \ref{thrmPt:1}.

Пусть $R$ --- соответствие
между $\mathbb{Z}$ и $\widetilde{\mathbb{R}}$, с искажением,
равным $1 + \epsilon$ и
в образе точки $i$ из $\mathbb{Z}$ лежит $(0,1)$. По \ref{lemDiamImage} диаметр
образа точки не может быть больше искажения соответствия,
следовательо образ $i$ лежит в $(-\epsilon, \epsilon) \cup \bigl\{(0,1)\bigr\}$. Это
означает, что для
$x$ не лежащих в $(-\epsilon, \epsilon)$, пара $(i, x)$ не лежит в $R$.
Обозначим за $\mathcal{N}$ множество всех целых чисел таких, что их
образ лежит в $(-\epsilon, \epsilon) \cup \bigl\{(0,1)\bigr\}$.
$\mathcal{N}$ не пусто и не равно $\mathbb{Z}$, следовательно, по лемме 3.1,
расстояние от $\mathbb{Z} \setminus\mathcal{N}$ до $\mathbb{R}$ будет не
меньше 1.
Из соответствия $R$ уберем пару $\bigl(i,(0,1)\bigr)$, а также
все пары $(k,x)$ такие, что $x \in  (-\epsilon, \epsilon)$.
Получившееся множество обозначим $R'$.
Так как все точки из $\mathbb{R}\setminus(-\epsilon, \epsilon)$ лежат в $R$
только в паре с точками из $ \mathbb{Z} \setminus\mathcal{N}$ и наоборот,
множество $R'$ будет соответствием между
$\mathbb{R}\setminus(-\epsilon, \epsilon)$ и
$ \mathbb{Z} \setminus\mathcal{N}$. Искажение
подмножества сответствия по определению не больше искажения
самого соответствия. Получаем цепочку неравенств:
\[1+\epsilon = \dis R \ge \dis R' \ge 2d_{GH}\bigl(\mathbb{R}\setminus(-\epsilon, \epsilon), \mathbb{Z} \setminus \mathcal{N}\bigr).\]
По неравенству треугольника
\[
    2d_{GH}\bigl(\mathbb{R}\setminus(-\epsilon, \epsilon), \mathbb{Z} \setminus \mathcal{N}\bigr) \ge
 2\Big|d_{GH}\bigl(\mathbb{R}, \mathbb{Z} \setminus \mathcal{N}\bigr) - d_{GH}\bigl(\mathbb{R}\setminus(-\epsilon, \epsilon), \mathbb{R}\bigr)\Big| \ge 2 - 2\epsilon.
\]

Получили неравество : $1 + \epsilon \ge 2 - 2\epsilon$.
Из него получаем нижнюю оценку на $\epsilon$:
\[\epsilon \ge \frac 1 3,\]
откуда $\dis R \ge \frac 4 3$ и
$d_{GH}\left(\widetilde{\mathbb{R}}, \mathbb{Z}\right) \ge \frac 2 3 > \frac 1 2$,
что доказывает \ref{thrmPt:2}.
\end{proof}
