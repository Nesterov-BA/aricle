\section{Мощность облаков}
Метрические пространства по своему определению являются множествами.
Соответственно для переноса конструкции расстояния Громова-Хаусдорфа на облака,
необходимо либо установить, что они --- множества, либо \\ соответствующим
образом изменить определение расстояния.
    \begin{theorem} Все облака представляют собой собственные классы.
	\end{theorem}
	\begin{proof} Для доказательства теоремы достаточно показать, что в любом
облаке лежат пространства сколь угодно большой мощности.  Пусть $X$ --
метрическое пространство мощности $\alpha$. Расширим это пространство до
пространства большей мощности. Обозначим $\Delta_\beta$ --- симплекс мощности
$\beta$, где $\beta > \alpha$. Обозначим $X_\beta = X \cup \Delta_\beta$.
Зафиксируем произвольную точку $x$ пространства $X$ и положим расстояние от нее
до любой точки симплекса равным $1$. Для точек $x' \in X$,
$y \in \Delta_\beta$ определим
$$\rho_{X_\beta}(y,x') = \rho_{X_\beta}(x',y) := \rho_X(x',x) + 1.$$
Расстояния между другими парами точек оставим без изменений.
Симметричность и неотрцательность расстояния $\rho_{X_{\beta}}$ очевидны.
Для того, чтобы
полученное расстояние являлось метрикой достаточно проверить выполнение
неравенства треугольника
$\rho_{X_\beta}(x',z') \le \rho_{X_\beta}(x',y') +\rho_{X_\beta}(y',z')$
только в том случае, если точки $x', y', z'$ не лежат одновременно в
$\Delta_\beta$ или в $X$. Случаи $x', z' \in \Delta_\beta$ и $ x', z' \in X$
очевидны. Разберем подробнее случаи, когда $x' \in X, z' \in \Delta_\beta$:
		$$ y' \in X: \rho_{X_\beta}(x', z') = \rho_X(x,x') + 1 \le \rho_X(x,y') + \rho_X(y',x') + 1 = \rho_X(x',y') + \rho_X(y',z')$$
		$$y' \in \Delta_\beta: \rho_{X_\beta}(x', z') = \rho_X(x,x') + 1 \le \rho_X(x',x) + 2 = \rho_X(x',y') + \rho_X(y',z')$$
		Итак, полученное пространство действительно будет метрическим. Осталось
заметить, что если вложить $X$ в $X_\beta$, то $X_\beta$ будет лежать в
замкнутой окрестности $X$ радиуса 1, что означает конечность расстояния между
ними.
 	 \end{proof}

 	 \begin{remark} Поскольку все облака являются собственными классами, между
любыми двумя облаками существует биекция. Это означает, в частности, что класс соответствий между любыми двумя облаками не пуст.
 	 \end{remark}

 	 \Def{Пусть $\mathcal{R}\big([X],[Y]\big)$ --- класс всех соответствий между
облаками $[X]$ и $[Y]$. Определим \emph{искажение} соответствия $\dis R$
аналогично \ref{defSootvet}. \emph{Расстоянием Громова--Хаусдорфа} между облаками
будем называть величину
$d_{GH}\big([X],[Y]\big) = \frac{1}{2}\inf\bigl\{\dis R : R\in \mathcal{R}\big([X],[Y]\big)\bigr\}$.}
\begin{theorem}
   Набор всех облаков не является множеством.
\end{theorem}
\begin{proof}
Пусть $n\Delta_\alpha$ --- симплекс мощности $\alpha$, умноженный на $n$. Рассмотрим дизъюнктное
объединение симплексов $A_\alpha = \bigsqcup_{n=1}^\infty n\Delta_\alpha$.
Зададим расстояние между точками $A_{\alpha}$ следующим образом:
\begin{itemize}
  \item Если $a, b$ лежат в одном симплексе $k\Delta_{\alpha}$, то $|ab| = k$.
  \item Если $a_{k}$ лежит в симплексе $k\Delta_{\alpha}$, а $a_{l}$ --- в $l\Delta_{\alpha}$,
        то $|a_{k} a_{l}| = \max(k,l)$.
\end{itemize}
Полученное расстояние будет метрикой и $A_\alpha$ --- метрическим пространством
мощности $\alpha$. Рассмотрим $A_\alpha$, $A_\beta$, $\beta>\alpha$ и
соответствие $R$ между ними. Для любого натурального $N$
мощность $|n\Delta_\beta|$ больше мощности $|A_\alpha|$. Это означает,
что найдутся  $a, b \in n\Delta_\beta$ и $x\in A_\alpha$такие, что
$(x,a), (x,b)$ лежат в соответствии $R$, и
$\big||xx| - |ab|\big| = n$. Следовательно $disR > n$ для всех натуральных
$n$, откуда
$d_{GH}(A_\alpha, A_\beta) = \infty$ для любых $\alpha, \beta$, не равных друг другу.
Это означает, что облака $\left[A_{\alpha}\right]$ и $\left[A_{\alpha}\right]$
различны, следовательно облаков не меньше, чем кардинальных чисел,
 которые не являются множеством.
\end{proof}
