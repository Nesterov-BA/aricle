%!TEX root = ../kursovaya.tex

\section{Подсчет расстояния между облаками $[\Delta_1]$ и $[\mathbb{R}]$}

Приведем лемму о расстоянии между облаками с пересекающимися стационарными группами.
\begin{lemma}
	Если два облака имеют нетривиальное пересечение стационарных групп,
	то расстояние между ними может быть равно $0$ или $\infty$.
\label{lemmaDist}
\end{lemma}
\begin{proof}
	Для любых облаков $[X], [Y]$ и любого $\lambds$ из $\mathbb{R}^{+}$ верно
\[\big|\lambda[X], \lambda[Y]\big| = \lambda\big|[X], [Y]\big|.\]
Отсюда, если $\lambda \neq 1$ лежит в стационарных группах обоих облаков, то
\[\big|[X],[Y]\big| = \big|\lambda[X], \lambda[Y]\big| = \lambda\big|[X], [Y]\big|.\]
Так как $\lambda \neq 1$, величина $\big|[X],[Y]\big|$ может быть равна
только $0$ или бесконечности.
\end{proof}
\begin{theorem} Пусть у облака $[Z]$ нетривиальная стационарная группа, и $Z$
является его центром. Также, пусть в этом облаке есть пространства $Y_{1}, Y_{2}$
такие, что $\max\big\{ |Y_{1},Z|, |Y_{2}, Z| \big\} = r>0$, а
$|Y_{1}, Y_{2}|>r$. Тогда, расстояние между облаками $[\Delta_1]$ и $[Z]$
равно бесконечности.
\label{thrmDist}
\end{theorem} 
\begin{proof} У облаков $[\Delta_1]$ и $[Z]$ стационарные группы имеют
нетривиальное пересечение, и, по лемме \ref{lemmaDist},
 расстояние между ними может быть
равно либо $0$, либо $\infty$. \\ Для доказательства утверждения теоремы
достаточно будет показать, что расстояние между ними не равно $0$.  Для этого
необходимо установить, что между ними не может существовать соответствия со
сколь угодно малым искажением. Итак, пусть $R$ --- соответствие между
$[\Delta_1]$ и $[Z]$, $\dis R = \epsilon < \infty$.
Зафиксируем $Y$ из $R(\Delta_1)$. По теореме \ref{thrmCenterImage} расстояние между $Y$ и $Z$ не
больше $2\epsilon$.\\ По условию теоремы выполнено неравенство:
 $$\max\big\{ |Y_{1},Z|, |Y_{2}, Z| \big\} = r < |Y_{1}, Y_{2}|$$
Неравенство означает, что существует $c > 0$ такое, что
$|Y_{1},Y_{2}| = (1 + c)r.$ \\ Вместе с $Y_{1}$ и
$ Y_{2}$ рассмотрим их прообразы $X_1 \in R^{-1}(Y_{1})$,
$ X_2 \in R^{-1}(Y_{2})$.  \\ Получаем следующую цепочку
неравенств:
	$$|X_1, \Delta_1| \le |Y_{1}, Y| + \epsilon \le |Y_{1}, \mathbb{R}| + |\mathbb{R}, Y| +\epsilon \le r + 2\epsilon + \epsilon = r + 3\epsilon.$$
	Аналогичное неравенство имеет место для $X_2$, при этом
	$$|X_1, X_2|  \ge |Y_{1}, Y_{2}| - \epsilon = (1+c)r - \epsilon.$$
	По замечанию \ref{remUltraMetric}:
	$$|X_1, X_2| \le \max\big\{ |X_1, \Delta_1|, |X_2, \Delta_1| \big\},$$
	$$\Updownarrow$$
	$$(1+c)r - \epsilon\le r + 3\epsilon,$$
	$$\Updownarrow$$
	$$\epsilon \ge \frac{cr}{4}.$$
	Мы получаем оценку снизу для $\epsilon = \dis R$. Это означает, что
искажение не может быть произвольно малым, и следовательно расстояние между
пространствами не может быть равно 0. Значит, оно равно бесконечности.
	
\end{proof}

Тем самым получаем следующее: любое облако с нетривиальной стационарной подгруппой
и не выполняющимся ультраметрическим неравенством для центра лежит
на бесконечном расстоянии от $[\Delta_{1}]$. В частности это верно
для облака $[\mathbb{R}]$.

\begin{corollary}
	В облаке $[\mathbb{R}]$ в качестве пространств $Y_{1}, Y_{2}$ можно взять
$\mathbb{Z}, \widetilde{\mathbb{R}}$. Для них, по теореме \ref{thrmRUltraMetric} будет выполнено
неравенство из условия теоремы
\ref{thrmDist}
 с $r = \frac 1 2$. Стационарная группа облака
$[\mathbb{R}]$ равна $\mathbb{R}^{+}$, то есть нетривиальна. Получаем, что расстояние между облаками
$[\Delta_{1}]$ и $[\mathbb{R}]$ равно бесконечности.
\end{corollary}
